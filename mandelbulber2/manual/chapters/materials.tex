\section{Materials}\label{materials}\index{materials}

\subsection{Defining and assigning materials}\label{defining-materials}\index{materials!defining}

A material defines the appearance of the surface and volume of an object and can be used to give an object an interesting or even a realistic look. A material which can be used in Mandelbulber can be just plain colors, colors generated by fractal properties, or colors from external files with textures. A material can also define physical properties of an object like transparency, reflectance, roughness or luminosity.

In the program materials can be defined as distinct entities which can be assigned to multiple objects. In the easiest scenario only one material could be used for all objects.

All defined materials are visible in the \emph{Materials} dock (figure \ref{materials_dock}).
From this window the user can manage the materials.

\simpleImageWithCaptionHalfWidth{img/manual/media/material_dock.png}
{Docked window with defined material}
{materials_dock}{h}

Clicking on a chosen material will bring up the \emph{Material editor}, where all material properties are visible.

The following options are available in this dock:

\begin{description}
		\item[New material] -- add new definition of a material
		\item[Delete material] -- delete selected material
		\item[Edit material] -- open separate window with \emph{Material editor}
		\item[Load material] -- open file selector for loading an already defined material
		\item[Save material as] -- open file selector for saving a material into a file
\end{description}

A material can be assigned to different types of objects: main fractal, boolean fractals or primitives. One material can me assigned to many objects. For every object there needs to be one assigned material.

\subsubsection{Assigning material to main fractal}\label{assigning-material-to-main-fractal}

To assign a material to the main fractal object go to \emph{Objects} dock and \emph{Global parameters} tab (figure \ref{materials_assign_to_main_fractal}). Clicking on the material icon will bring up the \emph{Material manager} window where the material can be selected. This selected material will be associated with the main fractal object. Further changes of the material properties in \emph{Material editor} will directly affect all objects with this assigned material.
\nopagebreak
\simpleImageWithCaptionHalfWidth{img/manual/media/material_assign_to_main_fractal.png}
{Material icon in Global parameters}
{materials_assign_to_main_fractal}{h}

\subsubsection{Assigning material to primitive object}\label{assigning-material-to-primitive}

To assign a material to a particular primitive object go to \emph{Objects} dock and \emph{Primitives} tab. Below the parameters of this primitive there is an icon for the material selection (figure \ref{materials_assign_to_primitive}).
 
\nopagebreak
\simpleImageWithCaptionHalfWidth{img/manual/media/material_assign_to_primitive.png}
{Material icon in definition of primitive object}
{materials_assign_to_primitive}{H}

\subsubsection{Assigning material to one of the fractals in boolean mode}\label{assigning-material-to-boolean}

To assign a material to a particular fractal when the boolean mode is enabled, go to \emph{Objects} dock and \emph{Fractals} tab. Then go to the formula tab of the fractal, which should be assigned with the material. The selection of the material is located below the fractal formula parameters (figure \ref{materials_assign_to_boolean}).

\nopagebreak
\simpleImageWithCaptionHalfWidth{img/manual/media/material_assign_to_boolean.png}
{Material icon below fractal formula parameters in boolean mode}
{materials_assign_to_boolean}{H}

\subsection{Editing parameters of materials}\label{materials-parameters}\index{materials!parameters}

The parameters of a material can be edited in the \emph{Material editor} dock. On the top of the material editor is a real-time preview of the material and the name of the material. The name can be edited freely for easier identification of the material.

\subsection{Gradients}\label{materials-gradients}

Gradients can be applied to fractal objects. If material with gradients is assigned to a primitive object, then gradient will be ignored and the plain color is used instead (parameter \emph{Surface color}) -- see figure \ref{materials_gradient_on_primitive}

\simpleImageWithCaptionHalfWidth{img/manual/media/material_gradients_on_primitives.png}
{Gradient applied to fractal and primitive objects}
{materials_gradient_on_primitive}{h}

\subsubsection{Gradient editor}\label{materials-gradient-editor}

For every parameter which uses a gradient there is the same type of gradient editor. A Gradient can have up to 100 intermediate colors. The first and the last color are always the same, because the gradient is repeated in a loop.

\simpleImageWithCaptionHalfWidth{img/manual/media/material_gradient_editor.png}
{Gradient editor}
{materials_gradient_edtor}{h}

The gradient editor has the following buttons (figure \ref{materials_gradient_edtor}):\index{materials!gradient editor}
\nopagebreak
\begin{description}
	\item[Defined colors (1)] -- colors used in the gradient. A double click on the color opens a window where the color can be changed. The color box can be moved along the gradient (only the first and last color cannot be moved)
	\item[Randomize colors (2)] -- randomizes all colors in the gradient
	\item[Randomize all (3)] -- randomizes the number of colors, their positions and the colors
	\item[Increase brightness (4)] -- increases brightness of all colors
	\item[Decrease brightness (5)] -- decreases brightness of all colors
	\item[Invert (6)] -- inverts all colors
	\item[Increase saturation (7)] -- increases saturation of all colors
	\item[Decrease saturation (8)] -- decreases saturation of all colors
\end{description}

Right click on the gradient opens the context menu which has the following options:
\nopagebreak
\begin{description}
	\item[Add] -- Adds new color box in the position of mouse pointer
	\item[Remove] -- Removes color box in the position of mouse pointer
	\item[Delete all] -- deletes all colors from the gradient. Only the first and last color are kept.
	\item[Change number of colors] -- changes number of colors with recalculation of color boxes
	\item[Grab colors from image] -- opens file selector to choose an image from which colors will be extracted
	\item[Load colors from file] -- opens file selector to choose a text file with an already defined gradient
	\item[Save colors to file] -- opens file selector where gradient will be saved as a text file.
	\item[Copy] -- copies gradient to clipboard. This can be used to copy all colors from one gradient to another one (e.g. to use the same colors for transparency as used for surface color)
	\item[Paste] -- pastes gradient from clipboard. 
\end{description}

\subsubsection{Gradient common options}\label{materials-gradient-common-options}

Gradients are used in the material when option \emph{Use colors from gradients} is enabled. Gradients can be used for the following channels: surface color, specular highlights, diffuse, luminosity, roughness, reflectance and transparency.

All of the channels have the following common settings:
\begin{description}
	\item[Palette offset] -- shift colors along the color palette (figures \ref{material-palette-offset-0}, \ref{material-palette-offset-02}, \ref{material-palette-offset-05})
	
	\threeImagesWithTwoCaptionsFullWidth
	{img/manual/media/material_palette_offset_0.png}
	{palette offset = 0}
	{material-palette-offset-0}
	{img/manual/media/material_palette_offset_02.png}
	{palette offset = 0.2}
	{material-palette-offset-02}
	{img/manual/media/material_palette_offset_05.png}
	{palette offset = 0.5}
	{material-palette-offset-05}
	
	\item[Color speed] -- frequency of color changing (figures \ref{material-color_speed_025}, \ref{material-color_speed_1}, \ref{material-color_speed_4})
	
	\threeImagesWithTwoCaptionsFullWidth
	{img/manual/media/material_color_speed_025.png}
	{color speed = 0.25}
	{material-color_speed_025}
	{img/manual/media/material_color_speed_1.png}
	{color speed = 1}
	{material-color_speed_1}
	{img/manual/media/material_color_speed_4.png}
	{color speed = 4}
	{material-color_speed_4}
\end{description}
	
\subsubsection{Coloring orbit trap algorithms}\label{materials-coloring-0rbi-trap-algorithms}
	
Option \emph{Coloring algorithm} allows selection of the algorithm which will be used to distribute colors. These algorithms use particular properties of the fractal calculation (results of orbit trap algorithms)\index{orbit trap}

\begin{description}
	\item[Standard] -- color index is calculated as a length of z vector
	\begin{center}
		\(c = |z|\)
	\end{center}
	\simpleImageWithCaptionHalfWidth{img/manual/media/material_color_algorithm_standard.png}
	{Standard coloring algorithm}
	{material-coloring-standard}{H}
	
	\item[orbit trap: z.Dot(point)] -- color index is calculated as a dot product of z vector and point coordinates
	\begin{center}
		\(c = |z \cdot p|\)
	\end{center}
	\simpleImageWithCaptionHalfWidth{img/manual/media/material_color_algorithm_zdotpoint.png}
	{orbit trap: z.Dot(point) coloring algorithm}
	{material-coloring-zdotpoint}{H}
	
	\pagebreak
	\item[orbit trap: Sphere] -- color index is calculated as distance of z vector from sphere of radius defined by parameter \emph{Orbit trap sphere radius}
	\begin{center}
		\(c = ||z - p|-r|\)
	\end{center}
	\simpleImageWithCaptionHalfWidth{img/manual/media/material_color_algorithm_sphere.png}
	{orbit trap: Sphere coloring algorithm}
	{material-coloring-sphere}{H}
	
	\item[orbit trap: Cross] -- color index is calculated as distance of z vector from orbit trap of cross shape
	\begin{center}
		\(c = \mathrm{min}(|z.x|, |z.y|, |z.z|)\)
	\end{center}
	\simpleImageWithCaptionHalfWidth{img/manual/media/material_color_algorithm_cross.png}
	{orbit trap: Cross coloring algorithm}
	{material-coloring-cross}{H}
	
	\pagebreak
	\item[orbit trap: Line] -- color index is calculated as distance of z vector from line defined by \emph{Orbit trap line direction vector}
	\begin{center}
		\(c = |z \cdot d|\)
	\end{center}
	\simpleImageWithCaptionHalfWidth{img/manual/media/material_color_algorithm_line_z1.png}
	{orbit trap: Line coloring algorithm (direction vector 0;0;1)}
	{material-coloring-line}{H}
\end{description}

\textbf{4D orbit trap color.} Enables 4D orbit trap algorithms. Result varies depending on the 4D fractal chosen and which orbit trap algorithm is used.

\textbf{Pre-V2.15 orbit trap color}. Enables backwards compatibility. In version 2.15 changes were made to make orbit trap calculations the same for both OpenCL and  Non-OpenCL.

\subsubsection{Gradient for surface color}\label{materials-surface_color-gradient}

Gradient for \emph{surface color} defines color channel of fractal surface. It is default way of coloring fractals.

\simpleImageWithCaptionHalfWidth{img/manual/media/material_color_algorithm_standard.png}
{Color gradient for fractal surface}
{material-color-gradient-example}{H}

\subsubsection{Gradient for specular highlights}\label{materials-specular-gradient}

Gradient for \emph{specular highlights} defines colors of specular effect. More details about this effects are in chapter \ref{materials-specular}

\simpleImageWithCaptionHalfWidth{img/manual/media/material_gradient_specular.png}
{Gradient for specular effect colors}
{material-color-gradient-specular}{H}

\subsubsection{Gradient for diffuse}\label{materials-diffuse-gradient}

Gradient for \emph{specular highlights} defines intensity of diffuse channel. Brighter gradient causes stronger light diffusion which means wider specular highlights and less visible reflections. More details about specular highlights are in chapter \ref{materials-specular}

\simpleImageWithCaptionHalfWidth{img/manual/media/material_gradient_diffuse.png}
{Gradient for diffusion channel}
{material-color-gradient-diffuse}{H}

\pagebreak
\subsubsection{Gradient for luminosity}\label{materials-luminosity-gradient}
Gradient for \emph{luminosity} defines intensity of luminosity channel. This property has effect when \emph{luminosity} parameter has value greater than zero. More details about luminosity are in chapter \ref{materials-luminosity}

\simpleImageWithCaptionHalfWidth{img/manual/media/material_gradient_luminosity.png}
{Gradient for luminosity channel with enabled Monte Carlo Global Illumination and luminosity = 10}
{material-color-gradient-luminosity}{H}

\subsubsection{Gradient for roughness}\label{materials-roughness-gradient}

Gradient for \emph{luminosity} defines intensity of roughness effect. This property has effect when \emph{Rough surface} is enabled. More details about roughness are in chapter \ref{materials-roughness}

\simpleImageWithCaptionHalfWidth{img/manual/media/material_gradient_roughness.png}
{Gradient for roughness intensity}
{material-color-gradient-roughness}{H}

\pagebreak
\subsubsection{Gradient for reflectance}\label{materials-reflectance-gradient}

Gradient for \emph{reflectance} defines amount and color of reflected light. This property has effect when \emph{Reflectance} parameter is greater than zero. More details about reflectance are in chapter \ref{materials-reflectance}

\simpleImageWithCaptionHalfWidth{img/manual/media/material_gradient_reflectance.png}
{Gradient for reflectance color and intensity}
{material-color-gradient-reflectance}{H}

\subsubsection{Gradient for transparency}\label{materials-transparency-gradient}

Gradient for \emph{transparency} defines color of transparency and opacity. This property has effect when \emph{Transparency} parameter is greater than zero. 
It recomended to use bright colors in the gradient for transparency. More details about transparency are in chapter \ref{materials-transparency}

\simpleImageWithCaptionHalfWidth{img/manual/media/material_gradient_transparency_bright.png}
{Gradient for transparency color - example of bright colors to get high transparency}
{material-color-gradient-transparency-bright}{H}

\simpleImageWithCaptionHalfWidth{img/manual/media/material_gradient_transparency.png}
{Gradient for transparency color}
{material-color-gradient-transparency}{H}

\subsection{Surface color}\label{materials-surface_color}

\subsection{Shading}\label{materials-shading}

\subsection{Specular highlights}\label{materials-specular}

\subsection{Luminosity}\label{materials-luminosity}

\subsection{Roughness}\label{materials-roughness}

\subsection{Reflectance}\label{materials-reflectance}

\subsection{Transparency}\label{materials-transparency}

\subsection{Common options for textures}\label{materials-textures}

\subsection{Advanced color parameters}\label{materials-advanced-color-parameters}

Coloring of a fractal can be derived from the information that is available throughout the iteration process.

The simplest is coloring a point based on the original location of the point.
Next is coloring a point based on the final location of the point at termination of the iteration process.

However we can derive more complex algorithms based on data obtained during the iteration process (e.g orbit trap algorithms and aux.color algorithms).

All of the above can be used together to form even more complicated algorithms.

\textbf{Orbit trap} algorithms have already been covered above, (however more complex versions can be coded but they will increase calculation time). The orbit trap output is called \textbf{colorMin}.

\textbf{Aux.color} value is generally generated by whether an if() condition is met on an iteration. For instance if z.x \textgreater\space folding\_limit then increase the aux.color value, if it does not then the value remains unchanged for that iteration. The final aux.color value for a point is the summation of the values at termination. This generally results in color banding. However some of the newer formulas have aux.color algorithms that do not produce color banding.

There are also a few transform based color algorithms, that use aux.color or aux.colorHybrid :

T>DIFS Hybrid Color - for coloring T>DIFS.

T>Hybrid Color - more options for coloring based on the points position at termination.

T>Hybrid Color2 - more options for coloring based on data from the iteration loop.

\subsubsection{Normal Mode Color(single formula mode and boolean mode}\label{materials-normal-mode-color}

For each formula there is one of five color functions assigned. The first three are simply the output from the orbit trap calculation multiplied by a constant.

\begin{lstlisting}
colorIndex = colorMin * constant;
\end{lstlisting}
	

a) case coloringFunctionDefault: constant = 5000; (the most commonly assigned).

b) case coloringFunctionIFS: constant = 1000; (menger, octo, sierpinski etc).

c) case coloringFunctionAmazingSurf: constant = 200; (only the first amazing surf formula).

msltoe Donut aux.color divided by number of iterations.

d) case coloringFunctionDonut: colorIndex = aux.color * 2000 / aux.i;

e) the fifth is case coloringFunctionAbox

This is calculated as the addition of three parts

\begin{lstlisting}
case coloringFunctionABox:
colorIndex = aux.color * 100 
+ r * defaultFractal->mandelbox.color.factorR / 1e13 
+ ((fractalColoring.coloringAlgorithm != fractalColoring_Standard) ? colorMin * 1000.0 : 0.0); 
\end{lstlisting}

aux.color is made up of components from the box fold and spherical fold.

radius at termination * parameter named absolute value of z  / 1e13 (mandelbox formula only and if in slot-1) Note: for converging fractals the radius at termination can be very small.


colorMin * constant, if coloring algorithm is Standard the constant = 0.0, otherwise the constant = 1000)


\subsubsection{Extra Hybrid Mode Color Options}\label{materials-extra-hybrid-mode-color-options}

When in hybrid mode (and more than one slot is enabled) the color is calculated as the addition of three parts:

\(final color = orbit trap  +  aux.color  +  \frac{radius}{aux.DE}\)

With the historic code the ratio of the three parts was fixed.

With Extra Hybrid Mode Color Options enabled, the influence of the three parts can now be controlled. This also allows for some backwards compatibility with Pre V2.15 color. If Extra Hybrid Mode Color Options is disabled then the calculation is run without the scales (this is faster).

\textbf{Notes: }The aux.color default value is 1.0, so even if the formula does not have any aux.color components there still be some influence. Depending on the formula, the \(\frac{radius}{aux.DE}\) component has little influence.

\subsubsection{Color by numbers}\label{materials-color-by-numbers}

Color by numbers is an exact mathematical approach (which also helps with fractal calculation diagnostics).
With the default palette gradient and the Color Speed set to 1.0, then a calculated colorValue of 2.0 will assign the color at a distance of two default palette intervals (pale yellow).
Color by numbers components are mixed by weights and the summation of the components is the Final ColorValue.

\textbf{Initial colorValue}. Default is 0.0, increasing this allows for the use of negative colorValues.

\textbf{ColorValue Initial Conditions Components}
By default all points have an initial colorValue of 0.0. Here it is possible to change the initial colorValue  based on the coordinates of the original point "c", (using radius\_c and c.x, c.y \& c.z.) Coloring can be applied using these functions alone or with other functions.

\textbf{Orbit trap component}. Apply a weight to the orbit trap output colorMin.

\textbf{auxillary color components}. Apply weights to aux.color and aux.colorHybrid outputs obtained from some fractals and transforms.

\textbf{radius components} 

A component value is added based on the distance of the point from the origin at termination.
\twoImagesWithTwoCaptionsFullWidth{img/manual/media/radius.png}
{Radius coloring component}
{radius}
{img/manual/media/radius_squared.png}
{Radius squared coloring component}
{radius_squared}{H}

\textbf{radius / DE components}

\subsubsection{Mandelbox coloring options}\label{materials-mandelbox}





