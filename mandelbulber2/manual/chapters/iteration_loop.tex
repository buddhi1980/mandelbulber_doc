\section{Iteration loop}\label{iteration-loop}

In section \ref{mandelbrot-set} there was mentioned that fractals are calculated by repeating of iteration loop. The loop is terminated when there is calculated number of iterations equals to \emph{maxiter} or bailout condition is achieved.
In this section will be explained what is calculated inside the loop.

\subsection{Single formula fractals}\label{single-formula-fractals}

The simplest 3D fractals are calculated using single fractal formula which is build from many equations and conditions. These equations can be modifications of Mandelbrot Set equation or can be different mathematical equations with several conditions.

Below there are 3 examples of fractals formulas with C language code

\subsubsection{Mandelbulb Power 2}\index{Mandelbulb}
\nopagebreak

This formula is modified Mandelbrot Set equation, expanded to \nth{3} dimension. Cross section at $ z_z = 0 $ looks exactly the same as Mandelbrot Set.
\nopagebreak

\begin{tabular}{l l}
	\includegraphics[width=0.3\linewidth]{img/manual/media/formula_mandelbulb_power_2}	
	& 
	\begin{minipage}[b]{0.5\linewidth}
		\begin{verbatim}[fontsize=\scriptsize]
double x2 = z.x * z.x;
double y2 = z.y * z.y;
double z2 = z.z * z.z;
double temp = 1.0 - z2 / (x2 + y2);
double newx = (x2 - y2) * temp;
double newy = 2.0 * z.x * z.y * temp;
double newz = -2.0 * z.z * sqrt(x2 + y2);
z.x = newx;
z.y = newy;
z.z = newz;
		\end{verbatim}
	\end{minipage}
\end{tabular} 

\subsubsection{Menger Sponge}\index{Menger Sponge}
\nopagebreak

This formula is Iterated Function System (IFS). It contains several transformations where some of them are conditional.
\nopagebreak

\begin{tabular}{l l}
	\includegraphics[width=0.3\linewidth]{img/manual/media/formula_menger_sponge.png}	
	& 
	\begin{minipage}[b]{0.5\linewidth}
		\begin{verbatim}[fontsize=\scriptsize]
z.x = fabs(z.x);
z.y = fabs(z.y);
z.z = fabs(z.z);
		
if (z.x - z.y < 0.0) swap(z.x, z.y);
if (z.x - z.z < 0.0) swap(z.x, z.z);
if (z.y - z.z < 0.0) swap(z.y, z.z);
		
z *= 3.0;
		
z.x -= 2.0;
z.y -= 2.0;
if (z.z > 1.0) z.z -= 2.0;
		\end{verbatim}
	\end{minipage}
\end{tabular} 

\subsubsection{Box Fold Bulb Pow 2}
\nopagebreak

This formula is a set of different transforms and equations. It's a good example which shows that fractal formula can be much more complicated than \emph{Mandelbrot Set}. 

First part is ``box fold''\index{transform!box fold} transform which do transformations based on box walls. Second part is ``spherical fold''\index{transform!spherical fold} which do transformations based on sphere. The end of formula is the same as \emph{Mandelbulb Power 2}. 
\nopagebreak

\begin{tabular}{l l}
	\includegraphics[width=0.3\linewidth]{img/manual/media/formula_box_fold_pwr2.png}	
	& 
	\begin{minipage}[b]{0.5\linewidth}
		\begin{verbatim}[fontsize=\scriptsize]
//box fold
if (fabs(z.x) > fractal->foldingIntPow.foldFactor)
z.x = sign(z.x) * fractal->foldingIntPow.foldFactor * 2.0 - z.x;
if (fabs(z.y) > fractal->foldingIntPow.foldFactor)
z.y = sign(z.y) * fractal->foldingIntPow.foldFactor * 2.0 - z.y;
if (fabs(z.z) > fractal->foldingIntPow.foldFactor)
z.z = sign(z.z) * fractal->foldingIntPow.foldFactor * 2.0 - z.z;

//spherical fold
double fR2_2 = 1.0;
double mR2_2 = 0.25;
double r2_2 = z.Dot(z);
double tglad_factor1_2 = fR2_2 / mR2_2;

if (r2_2 < mR2_2)
{
	z = z * tglad_factor1_2;
}
else if (r2_2 < fR2_2)
{
	double tglad_factor2_2 = fR2_2 / r2_2;
	z = z * tglad_factor2_2;
}

//Mandelbulb power 2
z = z * 2.0;
double x2 = z.x * z.x;
double y2 = z.y * z.y;
double z2 = z.z * z.z;
double temp = 1.0 - z2 / (x2 + y2);
zTemp.x = (x2 - y2) * temp;
zTemp.y = 2.0 * z.x * z.y * temp;
zTemp.z = -2.0 * z.z * sqrt(x2 + y2);
z = zTemp;
z.z *= fractal->foldingIntPow.zFactor;
		\end{verbatim}
	\end{minipage}
\end{tabular} 

\subsubsection{Processing of single formula fractals}

Single formula fractals are processed in simple way. Calculation of fractal formulas is repeated several times like it is showed on following diagrams:\nolinebreak
\nopagebreak

\includegraphics[width=\linewidth]{img/manual/media/iteration_loops.png}
	
When calculation of the iteration loop finished the result value of \emph{z} is used to estimate distance to fractal body and to calculate color of surface.

\subsection{Hybrid fractals}\index{fractal!hybrid}

There is possible to mix different fractal formulas by alternating them, to get new fractal shapes. That fractal are named \emph{hybrid fractals}. Mandelbulber program already have many different fractal formulas which gives opportunity to get very big variety of shapes. But using hybrid fractals increases possibilities a lot. 

\subsubsection{Iteration loop of hybrid fractals}

In general hybrid fractals are calculated in the same way as regular fractals. There is iteration loop, \emph{maxiter} and \emph{bailout} condition. The difference is in body of iteration loop. Instead of one fractal formula there can be used different fractal formula in every iteration. There is defined sequence which decides in which iteration the formula will be calculated. 
How the sequence will work depends on following selections:
\begin{itemize}
	\item Which fractal formulas are selected in formula slots
	\item How many iterations are assigned to each formula
	\item Range of iteration numbers where formula will be used
	\item From which fractal slot the sequence will be repeated
\end{itemize}

\includegraphics[width=\linewidth]{img/manual/media/iteration_loop_hybrid.png}

In Mandelbulber there is possible to define 9 different fractal formulas which ca be alternated. Each formula is configured in separate slot (tab)

\includegraphics[width=\linewidth]{img/manual/media/fractal_tabs.png}

By default there is possible do setup fractal formula in only first slot, because the program works in single fractal formula mode. There are two ways to enable hybrid fractals:
\begin{itemize}
	\item Click in any slot with number higher than one. The program will ask if you want to enable hybrid fractals or boolean mode. Select \emph{Enable hybrid fractals}
	\item Go to \emph{Objects} / \emph{Hybrid} tab. Tick \emph{Enable hybrid fractals} checkbox.
\end{itemize}

After that you can switch to any slot and define fractal formulas in each slot (if needed) like it's showed below

\includegraphics[width=\linewidth]{img/manual/media/fractal_tabs_with_defined_fractals.png}

There is selected \emph{Mandelbulb - Power 2} in slot \#1, \emph{Menger Sponge} in slot \#2 and \emph{Box Fold Bulb Pow 2} in slot \#3. These formulas will be used in next examples.

\subsubsection{One iteration for each slot}

The simplest way how hybrid fractal can be defined is to use each fractal formula for just only one iteration in looped sequence.

I example below, there is sequence of one \emph{Mandelbulb - Power 2}, one  and one \emph{Box Fold Bulb Pow 2}. Length of the sequence is 3, so by every 3 iterations will be repeated in the same way.

\includegraphics[width=\linewidth]{img/manual/media/iteration_loop_hybrid_sequence_1.png}

This sequence gives following shape which is a mix of properties of all 3 formulas.
\nopagebreak

\includegraphics[width=0.7\linewidth]{img/manual/media/hybrid_sequence_example_1.png}

Because in zero iteration is used \emph{Mandelbulb - Power 2}, the general shape of the fractal is a little similar to \emph{Mandelbulb - Power 2}

In next iteration there is used \emph{Menger Sponge} formula. Single iteration of this formula produces this shape:

\includegraphics[width=0.2\linewidth]{img/manual/media/single_iteration_of_menger_sponge.png}

Some properties of this shape are transfered to generated shape of hybrid fractal

\includegraphics[width=0.5\linewidth]{img/manual/media/single_iteration_of_menger_sponge_hybrid.png}

Menger sponge shape is distorted, because \emph{Mandelbulb - Power 2} has already deformed the space.

Third formula \emph{Box Fold Bulb Pow 2} adds leaf like features to the shape.

\includegraphics[width=0.4\linewidth]{img/manual/media/hybrid_sequence_example_1_leaf_shapes.png}

\subsubsection{More iterations for each slot}

\includegraphics[width=\linewidth]{img/manual/media/iteration_loop_hybrid_sequence_2.png}

\includegraphics[width=0.5\linewidth]{img/manual/media/hybrid_sequence_example_2.png}

\includegraphics[width=\linewidth]{img/manual/media/iteration_loop_hybrid_sequence_3.png}

\includegraphics[width=0.5\linewidth]{img/manual/media/hybrid_sequence_example_3.png}

\subsubsection{Range of iterations for slot}

\includegraphics[width=\linewidth]{img/manual/media/iteration_loop_hybrid_sequence_4.png}

\includegraphics[width=0.5\linewidth]{img/manual/media/hybrid_sequence_example_4.png}

\subsubsection{Changed order in sequence}

\includegraphics[width=\linewidth]{img/manual/media/iteration_loop_hybrid_sequence_5.png}

\includegraphics[width=0.5\linewidth]{img/manual/media/hybrid_sequence_example_5.png}
