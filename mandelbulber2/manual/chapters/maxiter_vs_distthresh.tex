\section{Ray-marching - Maximum number of iterations versus distance threshold
	condition}\label{ray-marching---maximum-number-of-iterations-versus-distance-threshold-condition}

The \emph{ray marching distance threshold}\index{ray marching!distance threshold} is the condition where a photon
marching along the ray comes within a specified distance from the fractal
surface and the ray-marching stops. This controls the size of the detail in the
image, and is normally set to vary such that greater detail is obtained for the
surface closest to the camera, and in the further regions of the fractal, the
distance threshold will be larger such that only bigger details are visible.
Enabling \emph{Constant Detail Size}\index{ray marching!constant detail size} on the \emph{Rendering Engine} tab will
make the distance threshold uniform.

There are two modes of stopping the ray-marching of each image pixel.

1st case: Stop ray-marching at distance threshold (\emph{Stop at maximum
	iteration} is disabled).

2nd case: Stop ray-marching at point when a maximum number of iterations is
reached (\emph{Stop at maximum iteration} is enabled).

First important note: \emph{Stop at maximum iteration} doesn't control the
fractal iteration loop. It controls only ray-marching. The iteration loop always
runs to achieve Bailout\index{termination condition!bailout}, then if bailout is not reached the iteration stops at
Maxiter (see page \pageref{bailout-maxiter}).

On figure \ref{stop_raymarching_at_disttrhersh} ray-marching stops at distance threshold\index{ray marching!distance threshold}. In most cases the fractal iteration
loop stops on bailout condition\index{termination condition!bailout}, because away from surface it is not possible
to reach Maxiter. It makes rendering of fractals much faster.

On figure \ref{stop_raymarching_at_maxiter} ray-marching stops at the photon step when the maximum number of iterations is
reached\index{termination condition!maxiter} (ray-marching distance threshold is ignored). In many cases iteration
loop stops on bailout condition (away from fractal surface), but on the fractal
surface the maximum number of iterations is calculated (when bailout is not
reached).

Figure \ref{stop-raymarching-at-bailout-low-maxiter} shows what happens if maximum number of iterations is set to 4. Even if Maxiter\index{termination condition!maxiter} is reached, the
ray-marching is continued until the ray marching distance threshold is reached.

\twoImagesWithTwoCaptionsFullWidth{img/manual/media/stop_raymarching_at_disttrhersh.png}
{Example for 1st case - stop ray-marching at distance threshold}
{stop_raymarching_at_disttrhersh}
{img/manual/media/stop_raymarching_at_maxiter.png}
{Example for 2nd case: Stop ray-marching at Maxiter}
{stop_raymarching_at_maxiter}{H}

Figure \ref{stop ray-marching-at-maxiter-low-maxiter} shows case when maximum number of iterations is reached. Ray-marching is stopped even
if distance threshold is not reached.

\twoImagesWithTwoCaptionsFullWidth{img/manual/media/stop_raymarching_at_disttrhersh_iter4.png}
{Example for 1st case: Stop ray-marching at bailout with low Maxiter}
{stop-raymarching-at-bailout-low-maxiter}
{img/manual/media/stop_raymarching_at_maxiter_iter4.png}
{Example for 2nd case: Stop ray-marching at maxiter with low maxiter}
{stop ray-marching-at-maxiter-low-maxiter}{h}

Enabling \emph{Constant Detail Size}\index{ray marching!constant detail size} on the \emph{Rendering Engine} tab will make the distance threshold uniform. This takes longer, but gives a more accurate representation of detail in the distance, and can be used to address color variation over distance (figure \ref{constant_detail_size}).

\simpleImageWithCaptionHalfWidth{img/manual/media/constant_detail_size.jpg}
{Constant detail size}
{constant_detail_size}{h}


